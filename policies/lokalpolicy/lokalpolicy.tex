\documentclass[11pt, includeaddress]{../../classes/cthit}
\usepackage{titlesec}

\titleformat{\paragraph}[hang]{\normalfont\normalsize\bfseries}{\theparagraph}{1em}{}
\titlespacing*{\paragraph}{0pt}{3.25ex plus 1ex minus 0.2ex}{0.7em}

\graphicspath{ {../../images/} }

\begin{document}

\title{Lokalpolicy}
\approved{2013--02--28}
\maketitle

\thispagestyle{empty}

\newpage

\makeheadfoot%

%Rubriksnivådjup
\setcounter{tocdepth}{2}
%Sidnumreringsstart
\setcounter{page}{1}
\tableofcontents

\newpage

\section{Syfte}
Sektionens lokaler är till för medlemmar i Teknologsektionen Informationsteknik.
\HUBBEN{} är i första hand avsedd för studier och som lunchlokal. Det bör hållas en rimlig ljudnivå under studietid (8.00 - 17.00 på studiedagar). Under den tiden får heller ingen alkohol förtäras i lokalen. Vidare är \HUBBEN{} avsedd för arrangemang för sektionens medlemmar och skall vara ett ställe där alla medlemmar kan umgås. Utöver detta dokument regleras lokalens nyttjande av dispositionsavtalet.

\begin{itemize}
	\item Anmärkning 1: Förutom IT­-sektionens egen lokalpolicy, lyder också sektionens medlemmar under kårens centrala lokalpolicy samt dispositionsavtalet mot Akademiska Hus. Se nedan: \\
	%MYhref definerad i classen, ger möjlighet att byta färg mm
	\MYhref{http://www.chs.chalmers.se/sites/default/files/uploads/Lokalpolicy.pdf}{Kårens lokalpolicy}\\
	\MYhref{http://styrit.chalmers.it/documents/files/Dispositionsavtal\%20Informationsteknik\%202012.pdf}{Dispositionsavtal}
	\item Anmärkning 2: Tentaveckan räknas från första dag i tentaveckan kl 08.00 till sista dag i tentaveckan kl 13.00. 
\end{itemize}


\section{Förteckning}
Sektionens lokaler består av:
\begin{itemize}  
  \item \HUBBEN{} 
  	\begin{itemize}  
	  \item Stora rummet
	  \item Köket
	  \item Grupprummet
	  \item Studierummet
	  \item Föreningsrummet
	  \item Bunkern
	\end{itemize}
\end{itemize}

\subsection{Grupprummet}
Grupprummet är främst avsett för individuella studier eller grupparbeten. Under luncher och efter klockan 17:00 kan grupprummet också fungera som möteslokal för sektionens olika organ. Ljudnivån ska vara studievänlig under studietid.

\subsection{Studierummet}
Studierrummet är främst avsett för studier och ska alltid vara tillgängligt för alla sektionsmedlemmar (undantag kan göras vid större arrangemang eller vid godkännande av styrelsen). Här ska det hållas en låg ljudnivå. Ingen alkohol får förekomma härinne. Ej heller under större arrangemang då studierummet enligt dispositionsavtalet ej tillåter fest.

\subsection{Föreningsrummet}
Föreningsrummet är endast avsett som förvaringsutrymme för sektionens olika organ.

\section{Tillträde}
\begin{itemize}
	\item Alla sektionsmedlemmar har tillträde till \HUBBEN{} såvida ej serveringstillstånd gäller. Om \HUBBEN{} är bokad bör detta dock respekteras.
	\item I undantagsfall har styrelsen eller lokalansvarig kommitté rätt att frånta enskild sektionsmedlem tillträde till \HUBBEN{}.
	\item Det är inte under några omständigheter tillåtet att sova i \HUBBEN{}.
\end{itemize}


\section{Hyra}
\subsection{Lokaler}
\HUBBEN{} får ej hyras ut.

\subsection{Inventarier}
Hubbens inventarier får hyras ut i samband med att \HUBBEN{} lånas ut till extern part.

\section{Boka}

\begin{itemize}
\item Bokningar får endast göras för sektionsnyttiga ändamål.
\item Styrelsen samt lokalansvarig kommitté har rätt att häva bokningar om de ej anses lämpliga. 
\end{itemize}

\subsection{\HUBBEN{}}
\HUBBEN{} får bokas av IT-­sektionens förtroendevalda efter kl 17.00 på läsdagar och dygnet runt i övrigt. Undantag gäller för tentaveckor (även omtentaveckor), då \HUBBEN{} inte får bokas annat än till studierelaterade arrangemang, samt under mottagningen då \HUBBEN{} är bokningsbar även under dagtid. Grupprummet bokas separat från övriga \HUBBEN{}. 

\subsection{Grupprummet}
Grupprummet får bokas under lunchtid och efter kl 17.00 under läsdagar och dygnet runt i övrigt. Undantag gäller för tentaveckor, då grupprummet inte för bokas annat än till pluggrelaterade arrangemang.

\subsection{Studierummet}
Studierummet får ej bokas.

\section{Inventarier}
Hubbens inventarier får gratis nyttjas av sektionsmedlemmarna då dessa har bokat \HUBBEN{}. Inventarier som införskaffas till \HUBBEN{} måste vara flamsäkra. Hubbens inventarier får ej lämna lokalen utan medgivande från lokalansvarig kommitté.

\section{Ansvar vid arrangemang}
\begin{itemize}
	\item Arrangör är skyldig att vara väl införstådd med detta dokument, lokalens dispositionsavtal samt studentkårens lokalpolicy. 
	\item Arrangör är ansvarig för skador som uppstår under arrangemang. 
	\item Arrangör är ansvarig för att återställa lokalen till ett acceptabelt skick efter arrangemang. Denna nivå sätts i samspråk med lokalansvarig kommitté vid utlämning.
\end{itemize}


\end{document}